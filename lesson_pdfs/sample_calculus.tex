\documentclass{article}
\usepackage[utf8]{inputenc}
\usepackage{amsmath}
\usepackage{amsfonts}
\usepackage{amssymb}
\usepackage{geometry}
\geometry{margin=1in}

\title{Introduction to Derivatives in Calculus}
\author{}
\date{}

\begin{document}

\maketitle

\section{What is a Derivative?}

A derivative represents the instantaneous rate of change of a function with respect to its variable. In simpler terms, it tells us how quickly a function is changing at any given point.

\subsection{Mathematical Definition}

The derivative of a function $f(x)$ is defined as:

\begin{equation}
f'(x) = \lim_{h \to 0} \frac{f(x+h) - f(x)}{h}
\end{equation}

This limit represents the slope of the tangent line to the curve at point $x$.

\subsection{Example: Power Rule}

For a simple function like $f(x) = x^2$:
\begin{itemize}
    \item The derivative is $f'(x) = 2x$
    \item This means at any point $x$, the slope of the curve is $2x$
    \item At $x = 3$, the slope is $6$
    \item At $x = 0$, the slope is $0$ (horizontal tangent)
\end{itemize}

\subsection{Real-World Applications}

\begin{enumerate}
    \item \textbf{Velocity}: If position is $s(t)$, then velocity is $s'(t)$
    \item \textbf{Acceleration}: If velocity is $v(t)$, then acceleration is $v'(t)$
    \item \textbf{Optimization}: Finding maximum and minimum values
    \item \textbf{Economics}: Marginal cost and revenue analysis
\end{enumerate}

\subsection{Practice Problems}

\begin{enumerate}
    \item Find the derivative of $f(x) = 3x^3 + 2x^2 - 5x + 1$
    \item What is the slope of $y = x^2$ at the point $(2, 4)$?
    \item If the position of an object is $s(t) = 16t^2$, find its velocity at $t = 3$ seconds.
\end{enumerate}

This foundational concept opens the door to understanding rates of change in all areas of science and mathematics.

\end{document}